\documentclass[conference]{IEEEtran}
\usepackage{cite}
\usepackage{graphicx}
% \usepackage{amsmath} %\usepackage{amssymb}
\usepackage{amsmath,amssymb,amsthm}
%\usepackage{algorithmic}
\usepackage{fixltx2e}
\usepackage{stfloats}
\usepackage{url}
\usepackage{hyperref}
\usepackage{xcolor}

\usepackage[justification=centering,font=small]{caption}
\usepackage{movie15}
\usepackage[normalem]{ulem}
\usepackage{bm}
\useunder{\uline}{\ul}{}
\newcommand{\norm}[1]{\left\lVert#1\right\rVert}
% for theorems &Co
\usepackage[utf8]{inputenc}
\newtheorem{theorem}{Theorem}[section]
\newtheorem{corollary}{Corollary}[theorem]
\newtheorem{lemma}[theorem]{Lemma}

\begin{document}
%
% paper title
% Titles are generally capitalized except for words such as a, an, and, as,
% at, but, by, for, in, nor, of, on, or, the, to and up, which are usually
% not capitalized unless they are the first or last word of the title.
% Linebreaks \\ can be used within to get better formatting as desired.
% Do not put math or special symbols in the title.
\title{Darcy flow:\\ Pollution of a drinking water reservoir}

% author names and affiliations
% use a multiple column layout for up to three different
% affiliations
\author{
\IEEEauthorblockN{Zampieri Luca}
\IEEEauthorblockA{Computational Science \& Engineering\\
\href{mailto:luca.zampieri@epfl.ch}{Email: luca.zampieri@epfl.ch}}
}

\maketitle
\vspace{-10cm}

\begin{abstract}
This paper reports the solutions of a problem of drinking water pollution using Finite Elements methods. In particular, since it involves the Darcy Equations, we will explore the use of the Raviart Thomas elements. The corresponding code implemented in FreeFem++ as well as the source of this report can be found on github: \url{https://github.com/LucaZampieri/Darcy_Flow}. This project is part of the course MA-468: Numerical Methods for Saddle Point Problems given by Prof. Annalisa Buffa at EPFL.

\end{abstract}
\IEEEpeerreviewmaketitle
\section{Notations}
We will use standard notation, complemented by the following (even if some are quite informal):
\begin{itemize}
\item $\norm{\cdot}_0 = \norm{\cdot}_{[L^2(\Omega)]^d}$, $\forall d$ number of dimensions
\item $\Gamma_{ij} = \Gamma_i \cup \Gamma_j$    with $ i,j < 10$
\item $H^1_0(\Omega)= \{ x \in H^1(\Omega) : x{\restriction}_{\partial \Omega}=0 \}$
\item $H^1_{0,\Gamma}(\Omega)= \{ x \in H^1(\Omega) : x{\restriction}_{\Gamma}=0 \}$
\item $H(div,\Omega) = \{ x \in H^1(\Omega) : div(x) = 0\}$
\item $H_{0}(div,\Omega) = \{ x \in H(div,\Omega) : x\cdot n{\restriction}_{\partial \Omega} = 0\}$
\item $H_{0,\Gamma}(div,\Omega) = \{ x \in H(div,\Omega) : x\cdot n{\restriction}_\Gamma = 0\}$
\item $\cdot^{CS}$ = use of Cauchy–Schwarz inequality
\item $\cdot^{HD}$ = use of the Hodler inequality
\item $\cdot^{SB}$ = use of the Sobolev inequality
\end{itemize}
if the domain of the space is omitted, it is assumed to be $\Omega$.


\section{Problem (question 1)}
For the complete statement of the problem, refer to the sheet given by the assistants. Here we make a quick reminder of the problem.\\
Consider a drinking water reservoir with fluid flowing left to right of the domain $\Omega$. There is a pollution accident at $C_2$ and a pump is placed at the center of the domain in $C_1$. We are interested in knowing how much pollutant has been extracted by the pump. \\
Considering $\Omega = [0,1]^2$ with border $\Gamma_i , i=1:4$ denoted in anti-clockwise sense, $C_1 = [0.48,0.52]^2 \in \Omega$ and $C_2=[0.28,0.32]^2 \in \Omega$. We then have to solve the equations:

\begin{equation}
    \begin{cases}
      -k \nabla p = \textbf{u} , & \text{in $\Omega$}  \\
      \nabla \cdot \textbf{u}, & \text{in $\Omega$} \\
      p=p_{in} , & \text{on $\Gamma_4$} \\
      p=p_{out} , & \text{on $\Gamma_2$} \\
      \textbf{u} \cdot \textbf{n} = 0, & \text{on } \Gamma_1 \cup \Gamma_3
    \end{cases}
    \label{eq:darcy}
\end{equation}
\begin{equation}
    \begin{cases}
      -\nu \Delta c + \textbf{u} \cdot \nabla c = g ,& \text{in $\Omega$}  \\
      c = 0, & \text{on } \Gamma_1 \cup \Gamma_3 \cup \Gamma_4 \\
      \nu \nabla c \cdot \textbf{n} = 0, & \text{on } \Gamma_2
    \end{cases}
    \label{eq:advection}
\end{equation}
Where (\ref{eq:darcy}) is a darcy equation, (\ref{eq:advection}) is an advection-diffusion equation and $f,g$ are defined as:
\begin{equation}
	f(\textbf{x})=
    \begin{cases}
      -1000,   & \text{if } \textbf{x} \in C_1 \\
      0,       & \text{if } \textbf{x} \notin C_1 \\
    \end{cases}
\end{equation}
and
\begin{equation}
	g(\textbf{x})=
    \begin{cases}
      1000, & \text{if } \textbf{x} \in C_2 \\
      0,    & \text{if } \textbf{x} \notin C_2 \\
    \end{cases}
\end{equation}
as such $f,g \in L^{\infty}(\Omega)$.\\
We fix $k=I$, $p_{in} = 2$ and $p_{out} = 0$. We are then interested in the quantity $\phi(c) = \int_{C_1} c$.

%%%%%%%%%%%%%%%%%%%%%%%%%%%% Weak Form %%%%%%%%%%%%%%%%%%%%%%%%%%%%%%%%%%%%%%%%%%%
\section{Weak forms (question 1)}
We want to prove the well-posedness of the system of equations. For this we rewrite the sytem of equations (\ref{eq:darcy},\ref{eq:advection} ) in the following form:\\
Find $\textbf{u} \in V$, $q \in Q$, $c \in W$ such that:  
\begin{equation}
    \begin{cases}
      a(u,v)+b(p,v) = F_a(v) , &\forall \textbf{v} \in V \\
      b(u,q) = F_b(q) ,        &\forall q \in Q \\
      a_{ad}(c,w) = G(w) ,     &\forall w \in W
    \end{cases}
    \label{eq:weak}
\end{equation}

where $a,b,a_{ad}$ are bilinear forms and $F_a, F_b, G$ are linear operators.


%%%%%%%%%%%%%%%%%%%%%%%%%%%%%%%   Darcy Weak   %%%%%%%%%%%%%%%%%%%%%%%%%%%%%%%%%
\subsection{Darcy equation}
For the Darcy equation we seek a mixed-form weak formulation. We set the spaces
$V=H(div,\Omega)$ and $Q=L^2(\Omega)$. Multiplying the first equation of (\ref{eq:darcy}) by a test function $v\in V$, the second equation by a test function $q\in Q$ and integrating we get:
\begin{equation}
    \begin{cases}
      \int_\Omega-k^{-1}\textbf{u}\cdot \textbf{v} 
      -\int_\Omega \nabla p \cdot \textbf{v} = 0 , & \forall \textbf{v} \in V  \\
      \int_\Omega (\nabla \cdot \textbf{u})q = \int_\Omega fq, & \forall q \in Q \\
    \end{cases}
    \label{eq:darcy_tmp}
\end{equation}
Integrating by parts the second term of the first equation of (\ref{eq:darcy_tmp}) and applying boundary conditions we get the mixed-form weak formulation:\\
Find \textbf{u} $\in V$, q $\in Q$ such that  :
\begin{equation}
    \begin{cases}
      \int_\Omega -k^{-1}\textbf{u}\cdot \textbf{v} + \int_\Omega  p (\nabla\cdot \textbf{v})
      = \int_{\partial\Omega}  \tilde{p} (\textbf{n}\cdot \textbf{v})  , & \forall \textbf{v} \in V  \\
      \int_\Omega (\nabla \cdot \textbf{u})q = \int_\Omega fq, & \forall q \in Q 
    \end{cases}
    \label{eq:darcy_weak}
\end{equation}
with $\tilde{p}$ is such that $\tilde{p}{\restriction}_{\Gamma_{13}} = 0 $, $\tilde{p}{\restriction}_{\Gamma_2} = p_{out} $ and $\tilde{p}{\restriction}_{\Gamma_4} = p_{in} $
Which is the form we desired (\ref{eq:weak}) since $a,b$ are clearly bilinear and $F_a,F_b$ are trivialy linear. \\
From the theory on mixed-forms, to prove well-posedness we additionally need:
\begin{itemize}
\item Coercivity on the Kernel i.e. $a(u_0,u_0) \geq \alpha \norm{u_0}_{H(div,\Omega)}^2$ , $\forall u_0 \in Ker(B)$ where $Ker(B)=\{v \in H_{0,\Gamma_{13}} : div(v)=0 \}$
\item Inf-Sup conditions i.e: $ inf_{q\in Q} sup_{v\in V} \frac{ b(v,q) }{ \norm{v}_V \norm{q}_Q } \geq \beta >0   $ 
\end{itemize}
We then proceed to prove these two properties:
%\subsubsection{Coercivity on the kernel}
\begin{proof}[proof: Coercivity on the kernel]
We have: 
$$a(u_0,u_0)=\int k^{-1} |u_0|^2 \geq k_{min} \norm{u_0}^2_{L^2(\Omega)} = \norm{u_0}^2_{H(div,\Omega)}$$ 
where $k_{min}$ is a minimal eigenvalue of $k$ and where we used both the facts that $k=I$ and that $u_0 \in Ker(B)$ thus $\norm{div(v)}_{L^2(\Omega)}=0$, which concludes the proof.
\end{proof}

%\subsubsection{Inf-Sup Condition}
\begin{proof}[Proof of Inf-Sup condition]
Consider the separate problem on the same domain:\\
Let $f \in L^2(\Omega)$:
\begin{equation}
	\begin{cases}
		-\Delta z = f_L ,  \text{  in } \Omega \\
		\partial_n z = 0 , \text{  on } \Gamma_{13}
	\end{cases}
	\label{eq:laplacian}
\end{equation}
which finds solution $z \in H^1(\Omega)$ with $\norm{z}_{H^1(\Omega)} \leq C_L \norm{f_L}_0 $.\\
If we now consider $\nabla z \in L^2(\Omega)^2$ of the previous problem, we have $\nabla z \in H_{0,\Gamma_{13}}(div,\Omega)$ and from the solution of the previous problem we get:
\begin{equation}
	\norm{\nabla z}_{H_{0,\Gamma_{13}}(div)} \leq C_L \norm{f_L}_0
	\label{eq:laplacianContinuity}
\end{equation} 
and shows that $B=div: H_{0,\Gamma_{13}}(\Omega) \longrightarrow L^2(\Omega)$ is surjective. \\
However, by the theorem of the equivalence of the Inf-Sup condition (\ref{th:equivalence}), this implies that the inf-sup condition holds.
\end{proof}
In parallel we could have done as below (\ref{proof:InfSupBis}).


\begin{proof}[Proof of Inf-Sup condition BIS]
\label{proof:InfSupBis}
We recall that $V=H_{0,\Gamma_{13}}(\Omega)$, $Q=L^2(\Omega)$. By integrating (\ref{eq:laplacian}) and multiplying by $q\in Q$ 
\begin{equation}
	\int_{\Omega} q \cdot div(\nabla z) \geq \norm{f_L}_0 \norm{q}_Q \geq
	\frac{1}{C_L} \norm{\nabla z}_V \norm{q}_Q
\end{equation}
Since $\nabla z, u \in V$ we can set $u = \nabla z$. Taking the supremum of the LHS we get:
\begin{equation}
	sup_{u\in V} \int_{\Omega} q \cdot div(u) \geq \norm{f_L}_0 \norm{q}_Q \geq 
	\frac{1}{C_L} \norm{u}_V \norm{q}_Q
\end{equation}
which can be rewritten as:
\begin{equation}
	sup_{u\in V} \frac{ \int_{\Omega} q \cdot div(u)}{\norm{u}_V} \geq \frac{1}{C_L} \norm{q}_Q
\end{equation}
which is equivalent to the Inf-Sup condition.
\end{proof}

We can finally conclude on the well-posedness of equation (\ref{eq:darcy_weak}) by the main theorem of the course.




%%%%%%%%%%%%%%%%%%%%% advec diffusion  %%%%%%%%%%%%%%%%%%%5
\subsection{Advection Diffusion}
Due to the previous subsection, we have $\textbf{u} \in V$. Set $W= H^1_{0,\Gamma_{134}} $. Multiplying the equation in (\ref{eq:advection}) by a test function $w \in W$, integrating and using integration by part we can reach the weak formulation:\\
Find $c \in W$ such that:
\begin{equation}
      a_{ad}(c,w) = \int_\Omega \nu \nabla c \nabla w 
      + \int_\Omega (\textbf{u} \cdot \nabla c) w
      = \int_\omega gw = G(w),  \forall w \in W  \\
    \label{eq:advection_weak}
\end{equation}
where we used the fact that $\int_{\partial\Omega} \nu (\nabla c \cdot \textbf{n}) w = 0$ since on $\Gamma_2$ we have homogeneous Neumann conditions and on the rest of $\partial\Omega$ we have $w=0$ since it belongs to $W$. We note that $a_{ad}(c,w)$ is a bilinear form and $G(w)$ a linear operator.\\
We want to conclude using the Lax-Milgram Lemma thus we desire:
\begin{itemize}
\item Continuity of G(w) which is implicit since $g\in L^{\infty}(\Omega)$
\item Continuity of the bilinear form $a_{ad}(c,w)$
\item Coercivity of the bilinear form $a_{ad}(c,w)$
\end{itemize}
Note that thanks to the Pointcare inequality (\ref{th:pointcare}), the norms $\norm{\cdot}_{H^1}$ and $\norm{\cdot}_{H^1_0}=|\cdot|_{H^1}=\norm{\nabla \cdot}_0$ are equivalent meaning there exist $m,M >0 $ s.t. $m\norm{\cdot}_{H^1} \leq \norm{\cdot}_{H^1_0} \leq M\norm{\cdot}_{H^1}$. \\
We can proceed:
\subsubsection{Continuity of the bilinear form} Rewriting
\begin{align}
	  |a_{ad}(c,w)| 
      & \leq^{CS} \nu \norm{\nabla c}_0 \norm{\nabla w}_0 + | \textbf{u} \cdot \nabla c w | \\
      & \leq^{HD} \nu \norm{\nabla c}_0 \norm{\nabla w}_0 + \norm{\textbf{u}}_{L^4(\Omega)} \norm{\nabla c}_0 \norm{w}_{L^4(\Omega)} \\
      & \leq^{SB} (\nu + \norm{\textbf{u}}_{L^4(\Omega)}) \norm{c}_{H^1} \norm{w}_{H^1}
\end{align}
and since $\norm{\cdot}_{H^1}$ and $\norm{\cdot}_{H^1_0}$ are equivalent norms, we have continuity.

\subsubsection{Coercivity of the bilinear form} 
\begin{equation}
      a_{ad}(c,c) = \nu \norm{\nabla c}_0^2 + \int_{\Omega} (\textbf{u}\cdot \nabla c) c
\end{equation}
To handle the last term of the equation, we need an assumption. The weakest we found is $\textbf{u} \cdot \textbf{n} \geq 0$. We can rewrite the last term as:
\begin{align}
      \int_{\Omega} (\textbf{u}\cdot \nabla c) c 
      & = \frac{1}{2} \int_{\Omega} \textbf{u}\cdot \nabla (c^2) \\
      & = - \int_{\Omega} (\nabla \cdot \textbf{u}) c^2 + \int_{\partial \Omega} (\textbf{u} \cdot \textbf{n}) c^2 \\
\end{align}
The first term vanishes due to $u\in V$ which is divergence free, and the second term vanishes on $\Gamma_{134}$ since it has diriclet boundary conditions.
Finally, using $\textbf{u} \cdot \textbf{n} \geq 0$:
\begin{equation}
      a_{ad}(c,c) \geq \nu \norm{\nabla c}_0^2 + \int_{\Gamma_2} (\textbf{u}\cdot \textbf{n}) c^2
      \geq \nu \norm{\nabla c}_0^2
\end{equation}
Which proves coercivity of $a_{ad}(\cdot,\cdot)$.\\



Using Lax-Milgram Lemma (\ref{th:laxMilgram}) we conclude on the well-posedness of the advection-diffusion equation (\ref{eq:advection_weak}) thus concluding on the well-posedness of the system of equations.


%%%%%%%%%%%%%%%%%%%%%%%%%%%%%%%%%%  Numerical discretization %%%%%%%%%%%%%%%%%%%%%%%%%%%%%%%%
\section{Numerical Discretization (Question 2)}
We should now find a suitable numerical discretization for the weak formulation of both the mixed-form of Darcy problem and the diffusion-advection equation for the pollutant agent, i.e. we want to find the spaces $V_h \subseteq V$, $Q_h \subseteq Q$ and  $W_h \subseteq W$ such that the discrete problem is well-posed. Finally, we will state the expected rate of convergence with these elements. \\ 

We will use results of \cite{Perotto}. To begin with let's discretize the domain. We consider $\tau_h$ as the triangulation og $\Omega$ (i.e. the cover with non-overlapping triangles of $\Omega$). We then define the discretized domain as:
\begin{equation}
	\Omega_h = int \bigg(\bigcup_{K\in \tau_h} K \bigg)
\end{equation}
and since $\Omega$ is a polygon the internal part of the union of the triangles of $\tau_h$ perfectly coincides with $\Omega$.\\

We define the space of polinomials of degree $r$ on $d$ dimensions
$$\mathbb{P}_r= \{p(\textbf{x}) = \sum\limits_{i_1+...+i_d \leq d} a_{i_1...i_d}\textbf{x}_1^{i_1}...\textbf{x}_d^{i_d}, a_{i_1...i_d} \in  \mathbb{R}\}$$
with $0 \leq i_1,...,i_d $.\\
Let us also define $\tilde{\mathbb{P}}_r$ as the space of polinomials of exact degree $r$: 
 

%%%%%%%%%%%%%%%%%%%%%%%%%%%%%%  Discrete Darcy %%%%%%%%%%%%%%%%%%%%%%%
\subsection{Discrete Darcy}
Lets begin with the Darcy equation. As seen during the lecture, a suitable family of finite elements space would be the Raviart-Thomas elements: 
\begin{equation}
	\mathbb{RT}_r (K) = \mathbb{P}_r^d(K) \oplus \textbf{x}\cdot \tilde{\mathbb{P}}_r(K)
\end{equation} 
where $ \tilde{\mathbb{P}}_r(K) = \mathbb{P}_r(K) \backslash \ \mathbb{P}_{r-1}(K) $. They are in fact (to some extension) the only appropiate elements to solve the Darcy Flow.\\
Setting $V_h=\{ \textbf{v} \in V : v|_k = \mathbb{RT}_r(K) , \forall K \in \tau_h \}$ and $Q_h= \{ q \in Q : d|_k \in  \mathbb{P}_r, \forall K \in \tau_h\}$. Note that $div(V_h) \subseteq Q_h$, which we need for consistency.\\
The discrete problem writes\\
Find $u_h \in V_h$, $p_h \in Q_h$ s.t.:
\begin{equation}
	\begin{cases}
		\int_{\Omega_h} k^{-1} u_h \cdot v_h + \int_{\Omega_h} div(v_h) p_h = ... ,&\forall v_h \in V_h \\
		\int_{\Omega_h} div(u_h) q_h = \int_{\Omega_h} f\ q_h, &\forall q_h \in Q_h
	\end{cases}
	\label{eq:DDarcy}
\end{equation}
and again, to prove well-posedness we need the two properties:
\begin{itemize}
\item $a(u_h,u_h)=\int_{\Omega_h} k^{-1} |u_h|^2 \geq \alpha \norm{u_h}^2_{V_h} , \forall u_h \in Ker(B_h)$
\item $sup_{u_h \in V_h} \frac{\int_{\Omega_h} div(u_h) p_h}{\norm{u_h}_{V_h}} \geq \beta \norm{p_h}_{Q_h}$
\end{itemize}
where $\norm{\cdot}_{V_h} = \norm{\cdot}_{V} $, $\norm{\cdot}_{Q_h} = \norm{\cdot}_{Q} $ and $Ker(B_h)=\{u_h \in V_h : \int_{\Omega_h} div(u_h) q_h = 0 \forall q_h \in Q_h \}$.\\
\subsubsection{Discrete coercivity on the kernel}
\begin{proof}
Take $q_h = div(u_h)$ (possible since $div(V_h)=Q_h$):
$$ \int_{\Omega_h} |div(u_h)|^2 = 0 \implies div(u_h) = 0 $$
and $Ker(B_h)\subset Ker(B)$.
Which implies $, \forall u_h \in Ker(B_h)$:
\begin{equation}
 a(u_h,u_h)=\int_{\tau_h} k^{-1} |u_h|^2 \geq \alpha (\norm{u_h}^2_0 + \norm{div(u_h)}^2_0)  
\end{equation}
since here $\norm{u_h}^2_0 = \norm{u_h}^2_0 + \norm{div(u_h)}^2_0 = \norm{u_h}^2_{V_h}$ as $\norm{div(u_h)}^2_0 = 0$
\end{proof}
\subsubsection{Discrete Inf-Sup conditions}
\begin{proof}
From the continuous Inf-Sup condition, with $p_h \in L^2(\Omega)$:
\begin{equation*}
	\exists u \in H(div,\Omega) : \frac{ \int_{\Omega} div(u)\cdot p_h}{\norm{u}_V} \geq  \norm{p_h}_0
\end{equation*}
by the definition of the $L^2$ projection $\Pi^s$ we have
\begin{align*}
	\int_{\Omega_h} div(u) p_h &= \int_{\Omega_h}\Pi^s(div(u))p_h \\
	&=\int_{\Omega_h} div(I_{RT} u) p_h \\
	&=\int_{\Omega_h} div(u_h) p_h
\end{align*}
and by the continuity of $I_{RT}$
\begin{equation*}
	\norm{I_{RT}u}_{H(div,\Omega)} \leq C_I \cdot \norm{u}_{H(div,\Omega)}
\end{equation*}
we can then write:
\begin{align*}
	 \frac{ \int_{\Omega_h} div(u_h)\cdot p_h}{\norm{u_h}_{H(div,\Omega)}}
	 &= \frac{ \int_{\Omega_h} div(u)\cdot p_h}{\norm{u_h}_{H(div,\Omega)}}
	 & \geq \frac{1}{C_I} \frac{ \int_{\Omega_h} div(u)\cdot p_h}{\norm{u}_{H(div,\Omega)}}
	 & \geq \frac{\beta}{C_I} \norm{p_h}_0
\end{align*}
proving that the Inf-Sup condition holds
\end{proof}

Thus the Discrete Darcy problem (\ref{eq:DDarcy}) is well-posed and verifies optimal error estimates in the energy norms:
\begin{align*}
	\norm{u-u_h}_V + \norm{p-p_h}_Q &\leq \\
	 C_{I} (inf_{u_h \in V_h}&\norm{u-vh}_V + inf_{q_h \in Q_h}\norm{p-p_h}_Q)
\end{align*}

%%%%%%%%%%%%%%%%%%%%%%%%%%%%%%  Discrete Advec-diff %%%%%%%%%%%%%%%%%%%%%%%
\subsection{Discrete Advection-Diffusion}
Setting $W_h=$, the discrete problem writes\\
Find $c_h \in W_h$ s.t.:
\begin{equation}
	  a_{ad}(c_h,w_h) = \int_{\Omega_h} \nu \nabla c_h \nabla w_h
      + \int_{\Omega_h} (u_h \cdot \nabla c_h) w_h
      = \int_{\Omega_h} g_h w_h = G(w_h)   	
\end{equation}
$\forall w_h \in W_h$.\\
Repeating the steps done for the continuous case, we use the Lax-Milgram Lemma to prove existence and uniqueness of the solution. Moreover, Cea's lemma gives us the error estimate:\\
$$ \norm{c-c_h}_W \leq \frac{M}{\alpha} inf_{w_h \in W_h} \norm{c-w_h}_W $$
where $M,\alpha$ are respectively the continuity and the coercivity contants of the bilinear form.

%%%%%%%%%%%%%%%%%%%%%%%%%%%%%%%%%% Rates of convergence %%%%%%%%%%%%%%%%%%%%%%%
\subsection{Rates of convergence}
Since we may need them for the convergence analysis, let's make few considerations on the grid. Setting $h_k=diam(K)$ where $diam(K)=max_{x,y\in K}|x-y|$, we will characterize the mesh size with $h=max_{K\in \tau_h} h_k$. For simplicity, in the following we may require that the grid satisfy a regularity condition:
\begin{equation}
	\frac{h_k}{\rho_k} \leq \delta , \forall K \in \tau_h
	\label{eq:regularMesh}
\end{equation}
for a suitable $\delta > 0$ and where $\rho_k$ is the sphericity of $K$ (i.e. the diameter of the circle inscribed in the triangle K). Note that regularity (\ref{eq:regularMesh}) usually excludes anisotropic grids. Since we will use anisotropic grids it is important to notice that the results we will present can be generalised to anisotropic grids with further work. Following \cite{Perotto} some of these generalisations can be found in \cite{anisotropic} and \cite{anisotropic2}.\\

By theorem 4.7 of \cite{Perotto} we have:
$$ \norm{c-c_h}_{H^1(\Omega)} \leq \frac{M}{\alpha} C_e h^r |c|_{H^{r+1}(\Omega)} $$
where $C_e$ is a constant independent of h and u, $M,\alpha$ are respectively the continuity and the coercivity constant of the bilinear form, and we have used the regularity of the mesh. \\
And from exercise session 13 we have:
\begin{align*}
	\norm{u-u_h}&_{H(div,\Omega)} + \norm{p-p_h}_{L^2(\Omega)} \leq  \\
	&C_d h^{r+1}
(|u|_{H^{r+1}(\Omega)}+|div(u)|_{H^{r+1}(\Omega)}+|p|_{H^{r+1}(\Omega)})
\end{align*}
where $C_d$ is a constant independent of $h$ and $u$.\\  

Assuming that the solutions are regular enough, for RT1,P1,P1 we expect thus to have linear convergence for $c$ and quadratic convergence for $u$ and $p$. In the following section we will investigate if this rates are respected.


%%%%%%%%%%%%%%%%%%%%%%%%%%%%%%%%%% RESULTS %%%%%%%%%%%%%%%%%%%%%%%%%%%%%%%%%%%%
\section{figures}

\begin{figure}[!ht]
\centering 
%\includegraphics[width=0.5\textwidth]{images/conv1D.jpg}
\caption{lala}
\label{fig:fig7}
\end{figure}

\begin{figure}[!ht]
\centering 
\begin{minipage}[t]{4cm} 
\centering 
%\includegraphics[width=4cm]{images/dilation.png}
\caption{im2} 
\label{fig:fig8}
\end{minipage} 
\begin{minipage}[t]{4cm} 
\centering 
%\includegraphics[width=4cm]{images/stride.png}
\caption{im1}
\label{fig:fig9}
\end{minipage} 
\end{figure}

%%%%%%%%%%%%%%%%%%%%%%%%%%%%%%%%%%  Appendix %%%%%%%%%%%%%%%%%%%%%%%%%%%%%%%%%%%%%%%%%%
\section{Appendix}\label{sec:appendix}
\subsection{Code}
code: 
\subsection{Properties, Lemmas \& Theorems}
Some of these theorems are in the course, the others can be found in \cite{Salsa} and \cite{Perotto}. \\
\begin{theorem}[Sobolev Embeddings\cite{Salsa}]
\label{th:sob}
Let $ u \in H^1(\mathbb{R}^2)$ \\
Then $u \in L^p(\mathbb{R}^2)$ for $2 \leq p < \infty$, and
$$\norm{u}_{L^p(\mathbb{R}^2)} \leq c(p) \norm{u}_{H^1(\mathbb{R}^2)} $$
\end{theorem}
\begin{lemma}[Lax-Milgram]
\label{th:laxMilgram}
Let V be a Hilbert space and $a(.,.)$ a bilinear form. If
$$ a(u,v) \leq M \norm{u}_V \norm{v}_V \qquad \qquad a(u,u)\geq \alpha \norm{u}_V^2$$
then the problem \textbf{Find u $\in$ V: a(u,v)=(f,v) $\forall$ v $\in$ V} admits a unique solution.
\end{lemma}

\begin{lemma}[Pointcare]
\label{th:pointcare}
If one of the following is true:
\begin{itemize}
\item $u \in H^1_{0,\Gamma}(\Omega)$ with $\Gamma \neq \emptyset$ \\
\item $u \in H^1(\Omega) $ and $u=0$ a.e. on some subset $E\subset \Omega $ with $|E|=a>0$ \\
\end{itemize}

then $\exists C_p$ independent of $u$ such that:
\begin{equation}
	\norm{u}_{L^2(\Omega) \leq C_p \norm{\nabla u}_{L^2(\Omega,\mathbb{R}^n)}}	
\end{equation}
\end{lemma}

%\begin{theorem}[Existence and uniqueness for mixed form]
%\label{th:mainTheorem}
%
%\end{theorem}

%\begin{theorem}[Equivalence for inf-sup Condition]
%\label{th:equivalence}
%\end{theorem}

\begin{thebibliography}{1}

\bibitem{Salsa}[PDE]
Partial Differential Equations in Action - From Modelling to Theory | Sandro Salsa | Springer.” Accessed June 7, 2018. https://www.springer.com/gp/book/9783319150932.

\bibitem{Perotto}[ANEDP]
“Numerical Models for Differential Problems | Alfio Quarteroni | Springer.” Accessed June 7, 2018. https://www.springer.com/gp/book/9788847058835.

\bibitem{anisotropic}
“Anisotropic Finite Elements: Local Estimates and Applications (Advances in Numerical Mathematics): Thomas Apel: 9783519027447: Amazon.Com: Books.” Accessed June 8, 2018. https://www.amazon.com/Anisotropic-Finite-Elements-Applications-Mathematics/dp/3519027445.

\bibitem{anisotropic2}
Formaggia, L., and S. Perotto. “New Anisotropic a Priori Error Estimates.” Numerische Mathematik 89, no. 4 (October 1, 2001): 641–67. https://doi.org/10.1007/s002110100273.



\end{thebibliography}

\end{document}